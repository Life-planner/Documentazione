\section{GitHub Repository and Deployment Info}
\label{secD4:GitHubRepositoryAndDeploymentInfo}
Riguardo alla descrizione del repo Git, che nel nostro caso si chiama "Codice" (repo dove è presente tutto il codice da noi sviluppato per l'implementazione del sito PlanIt, sia BackEnd che FrontEnd), molto è già descritto in \ref{apd:ProjectStructure}. Si sottolinea che abbiamo usato la stessa repo "Codice" per andare a sviluppare il FrontEnd e BackEnd, ma andando a creare due branch, di nome "FrontEnd" e "API" rispettivamente. In questa capitolo andiamo a descrivere altre cartelle presenti nel nostro git "Codice" (link: \href{https://github.com/Life-planner/Codice} {https://github.com/Life-planner/Codice}). Infatti in questa cartella git è presente anche la cartella "\_tests\_" dove sono presenti tutti i test effettuati per ciascuna API che è stata implementata, divisi a seconda di quale struttura dati (guardare Extract Diagram \ref{apd:ResourceExtraction}) riguardano. Inoltre, in  ciascuno file della cartella api, è presente la documentazione relativa alle funzioni presenti in tale file. E' stato fatto in questo modo, ovvero la presenza della documentazione nel relativo file, in quando abbiamo usato una libreria apposita di Swagger per NextJS che richiede tale procedura per andare a formare la documentazione. Infatti, grazie "next-swagger-doc", basta fornire la documentazione in ciascun file e automaticamente viene compilato un file "swagger.json" dove è scritta tutta la documentazione in formato ".json". Questo file è contenuto nella cartella "public", dove sono presenti anche tutti gli "asset" del progetto.\\
In questo capitolo presentiamo anche le informazioni riguardo al deployment e al link per eseguire il prototipo PlanIt da noi implementato. Per l'hosting del nostro sito abbiamo deciso di utilizzare il sito di hosting netlify. \\ Questi sono i link da noi fatti da cui si può usufruire di ciò che abbiamo implementato:
\begin{itemize}
    \item sito: \href{https://plan-it.it} {https://plan-it.it};
    \item documentazione: \href{https://plan-it.it/apidoc} {https://plan-it.it/apidoc};
    \item testing, prima opzione: \href{https://plan-it.it/test-report.html} {https://plan-it.it/test-report.html}
    \item testing, seconda opzione: \href{https://plan-it.it/coverage/index.html} {https://plan-it.it/coverage/index.html}

\end{itemize}
