\section*{Scopo del documento}
\addcontentsline{toc}{section}{Scopo del documento}

\todo{da sistemare, copia e incollato}
Il presente documento riporta tutte le informazioni necessarie per lo sviluppo di una parte dell'applicazione PlanIt. In particolare, presenta tutti gli artefatti necessari per realizzare i servizi di creazione e modifica di eventi, calendari e utente autenticato nel sito.
Partendo dalla descrizione degli user flows legati, soprattutto, al ruolo dell'utente autenticato dell'applicazione, il documento prosegue con la presentazione delle API necessarie (tramite l'API Model e il Modello delle risorse) per poter visualizzare, creare e modificare sia gli eventi che i calendari necessari per implementare l'applicativo PlanIt, che ricordiamo essere un applicativo di gestione calendari.
Per ogni API realizzata, oltre ad una descrizione delle funzionalità fornite, il documento presenta la sua documentazione e i test effettuati. Per i test effettuatati, verrà anche fornito un documento html con il resoconto di questi. Infine una sezione e' dedicata alle informazioni del Git Repository e il deployment dell'applicazione stessa. \\ Di seguito sono presenti gli argomenti che verranno trattati in questo documento.

\begin{itemize}
    \item \hyperref[secD4:UserFlows]{User Flows};
    \item \hyperref[secD4:ApplicationImplementationAndDocumentation]{Application Implementation and Documentation};
    \item \hyperref[secD4:APIDocumentation]{API Documentation};
    \item \hyperref[secD4:FrontEndImplementation]{FrontEnd Implementation};
    \item \hyperref[secD4:GitHubRepositoryAndDeploymentInfo]{GitHub Repository and DeploymentInfo};
    \item \hyperref[secD4:Testing]{Testing};
\end{itemize}
