\section*{Scopo del documento}
\addcontentsline{toc}{section}{Scopo del documento}
Il presente documento riporta la definizione dell'architettura del progetto PlanIt usando diagrammi delle classi in Unified Modeling Language (UML) e codice in Object Constraint Language (OCL). Nel precedente documento D2 è stato presentato il diagramma degli use case, il diagramma di contesto e quello dei componenti con le relative descrizioni e specificazioni. Ora, tenendo conto di questa progettazione, viene definita l'architettura del sistema dettagliando da un lato le classi che dovranno essere implementate a livello di codice e dall'altro la logica che regola il comportamento del software. Le classi vengono rappresentate tramite un diagramma delle classi in linguaggio UML. La logica viene descritta in OCL perché tali concetti non sono esprimibili in nessun altro modo formale nel contesto di UML. Tutti i diagrammi, che verranno presentati, hanno una descrizione a loro associati per delinearne il loro scopo e funzionalità.

\begin{itemize}
    \item \hyperref[secD3:DiagrammaDelleClassi]{diagramma delle classi};
    \item \hyperref[secD3:ObjectConstraintLanguage]{object constraint language};
    \item \hyperref[secD3:DiagrammaECodiceObjectConstraintLanguage]{diagramma delle classi e codice object constraint language}.
\end{itemize}

