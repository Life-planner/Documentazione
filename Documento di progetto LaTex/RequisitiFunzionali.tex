\section{Requisiti funzionali}
\label{sec:RequisitiFunzionali}

Nel seguente capitolo verranno presentati i requisiti funzionali (RF)
del sistema con la motivazione della loro presenza e legame 
con gli obiettivi sopra citati. 
\begin {mylist} {RF}
    \requisitiLink{rf.one}  \\ %macano i titoli
    Il sistema deve permettere all'utente di registrarsi sul sito (\hyperlink{ob.one}{OB1}) in qualsiasi momento 
    esso voglia. Ad ogni modo il sistema deve rendere disponibile all'utente di accedere al sito
    anche come utente anonimo. \\ \\
    \textbf{UTENTE AUTENTICATO}
    \requisitiLink{rf.two} {ciao}
    Il sistema, agli utenti autenticati sul sito, darà la possibilità di convidere 
    il proprio calendario su più dispositivi. Inoltre, grazie all'autenticazione, il sito
    presenterà un più alto livello di privacy e security, per preservare i propri dati sensibili..
    (RFNx). \\ \\
    \textbf{UTENTE NON AUTENTICATO}
    \requisitiLink{rf.three} \\
    L'accesso mediante utente anonimo non presenterà problemi dal punto di vista 
    della sicurezza e privacy (RFN *), ma andrà a discapito della condivisione
    tra dispositivi diversi.
    \requisitiLink{rf.four} \\
    Il sistema deve dare la possibilità all'utente, sia anonimo che non (RF1), di aggiungere
    un evento (impegno) al calendario mediante un form (pop-up). Nella compilazione
    dell'impegno il sistema deve permettere al cliente di imporre restrizioni, f.e. quando porre
    l'attività da svolgere. Il pop-up, inoltre, deve presentare dei form (non so, una 
    zona del pop-up) dove l'utente può impostare la priorità dell'impegno, una descrizione e un
    titolo di quest'ultimo.
    deve essere data la possibiltà di imporre impegni routine oppure impegni ripetuti su 
    più giorni. 
    \requisitiLink{rf.five} \\
    Una volta compilato il pop-up di aggiunta impegno, il sistema 
    deve inserire l'impegno nel calendario seguendo le restrizioni e priorità inserite 
    dall'utente a tempo di compilazione dell'evento (RF4).  
    \requisitiLink{rf.six} \\
    Il sistema deve presentare un form personalizzato per l'impostazione delle ore
    di sonno. Il sito deve dare la possibilità all'utente di poter modificare le ore 
    di sonno giorno per giorno, oppure applicare le stesse ore per più giorni (OBx)
    \requisitiLink{rf.seven} \\
    Il sistema deve inviare delle notifiche per ciascun impegno. Nel pop-up
    di compilazione dell'evento (RF6) deve essere data la possibilità all'utente
    di impostare una notifica personalizzata riguardo a tale attività.
    Le personalizzazioni riguardano: quando ricevere tale notifica e il titolo di questa, per 
    default quest'ultimo sarà il titolo dato all'evento.
    \requisitiLink{rf.eight} \\
    In quanto è un'applicazione molto diffusa tra gli utenti che utilizzano calendari,
    il sistema deve permettere all'utente di poter importare eventi da Google Calendar. 
    Quando gli eventi saranno aggiunti da Google Calendar, questi saranno
    imposti nel calendario di PlanIt seguendo le stesse regole per gli impegni 
    aggiunti direttamente dal sito (si veda RF5).
    \requisitiLink{rf.nine} \\
    Il sito deve presentare delle infografiche sull'uso tempo. L'utente, così, potrà
    visualizzare dei grafici esemplificativi di come il proprio tempo viene speso. 
    (OBx)
    \requisitiLink{rf.ten} \\
    Il sistema deve riorganizzare automaticamente il calendario in caso di ritardi sugli 
    impegni; sarà data la possibilità all'utente di notificare il sistema di ritardi e 
    a quel punto il sito dovrò ripianificare il calendario, sempre secondo le regole mostrate
    in RF5 (seguono sempre le stesse regole).  
    \requisitiLink{rf.eleven} \\
    Il sistema, a fine giornata, deve presentare un resoconto, dove l'utente potrà
    comunicare le attività fatte e non, in modo tale da dare la possibilità al sistema di 
    ricalcolare eventuali modifiche in base agli impegni non conclusi (si veda
    anche RF8 per modifiche calendario per ritardi)..
    



\end{mylist}