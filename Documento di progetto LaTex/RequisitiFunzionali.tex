\section{Requisiti funzionali}
\label{sec:RequisitiFunzionali}

Nel seguente capitolo verranno presentati i requisiti funzionali (RF) del sistema con la motivazione della loro presenza e legame con gli obiettivi sopra citati.

\begin{listaPersonale}{RF}
	\elemento[ACCESSO E REGISTRAZIONE AL SITO]{rf:1} Il sistema deve permettere all'utente di avere 3 livelli di accesso:
	\begin{itemize}
		\item utente autenticato-standard(\ref{ob:1});
		\item utente non autenticato (\ref{ob:1});
		\item utente autenticato-premium (\ref{ob:6});
	\end{itemize}
	Il sistema deve permettere all'utente di registrarsi sul sito (\ref{ob:1}) in qualsiasi momento esso voglia e inoltre deve permettere di abbonarsi al servizio premium, con un account preesistente o anche durante la fase di registrazione.\\
	A ogni modo il sistema deve rendere disponibile all'utente di utilizzare il sito anche come utente non autenticato in versione demo(\ref{rf:3}).

	\begin{listaPersonale2}{RF}
		\elemento[METODI AUTENTICAZIONE]{rf:1:1} L'autenticazione può essere fatta con email e password o con servizi di terze parti, più nello specifico con Google, Apple ed eventuali servizi di terze parti (\ref{ob:1}).
	\end{listaPersonale2}

	\elemento[UTENTE AUTENTICATO-STANDARD]{rf:2} Un utente autenticato standard è un utente registrato che ha effettuato il login e che usufruisce delle funzionalità base della piattaforma.

	\begin{listaPersonale2}{RF}
		\elemento[FUNZIONALITA']{rf:2:1} Il sistema offrirà agli utenti con account autenticato-standard tutti i servizi sottoelencati, eccetto quelli descritti nella sezione utente autenticato-premium (\ref{rf:2.2}), con le limitazioni elencate in \ref{rf:2.2.2}. Inoltre sarà possibile aggiungere o modificare il proprio username da usare per un più rapido riconoscimento nei calendari condivisi.

		\elemento[UTENTE AUTENTICATO-PREMIUM]{rf:2.2} L'utente avrà la possibilità anche di accedere a un maggior numero di funzionalità della piattaforma grazie alla sottoscrizione di un abbonamento mediante un pagamento mensile(\ref{ob:6}). Inoltre il pagamento degli utenti premium contribuirà alla sostenibilità economica del sito.

		\begin{listaPersonale3}{RF}
			\elemento[PAGAMENTO]{rf:2.2.1} Il sistema offrirà la possibilità all'utente di pagare l'abbonamento con un servizio di pagamento esterno. (\ref{ob:6})

			\elemento[FUNZIONALITA' AGGIUNTIVE(\ref{ob:6})]{rf:2.2.2} Grazie all'account premium, il sistema permetterà di accedere alle funzionalità standard e a dei servizi aggiuntivi:
			\begin{itemize}
				\item poter avere un numero illimitato di calendari personali in base allo loro scopo: gli utenti non premium potranno avere solo 5 calendari personali.
				\item poter avere un numero illimitato di calendari condivisi: gli utenti standard potranno averne 3.
			\end{itemize}

		\end{listaPersonale3}
	\end{listaPersonale2}

	\elemento[UTENTE NON AUTENTICATO]{rf:3} Nel caso di un utente non autenticato (\ref{ob:1}), il sistema permetterà di effettuare l'accesso e/o registrazione nelle modalità descritte in \ref{rf:1}. Nel caso l'utente non volesse registrarsi o accedere, il sito darà la possibilità di usufruire delle funzionalità in versione demo.

	\begin{listaPersonale2}{RF}
		\elemento[DEMO]{rf:3.1}	L'utente non autenticato avrà la possibilità di utilizzare il sito senza che le modifiche fatte vengano salvate sui server della piattaforma; nel caso in cui l'utente si registrasse sul sito, verranno caricate e rese disponibili su più dispositivi.

		\begin{listaPersonale3}{}
			\elemento[FUNZIONALITÀ MANCANTI]{rf:3.1.1} L'utente non autenticato non avrà a disposizione diverse funzionalità tra cui \ref{rf:4}, \ref{rf:8}, \ref{rf:10}, \ref{rf:11}.
		\end{listaPersonale3}
	\end{listaPersonale2}

	\elemento[CONDIVISIONE DEL CALENDARIO]{rf:4} Il sistema darà la possibilità di condividere il proprio calendario su più dispositivi (\ref{ob:8}).

	\begin{listaPersonale2}{RF}
		\elemento[INTERAZIONE DI CALENDARI CON ALTRE PERSONE]{rf:4.1} Il sistema deve offrire la possibilità agli utenti autenticati d'interagire tra loro mediante la condivisione e integrazione dei loro calendari (\ref{ob:8}). Grazie alla condivisione e integrazione dei calendari tra gli utenti del sito, i clienti avranno la possibilità di programmare gli eventi in comune.
	\end{listaPersonale2}

	\elemento[COMPILAZIONE/MODIFICA EVENTO]{rf:5} Il sistema deve dare la possibilità all'utente di aggiungere e modificare un evento del calendario mediante l'utilizzo di un pop-up (\ref{ob:2}). Nella compilazione dell'attività, il sistema deve permettere al cliente:
	\begin{listaPersonale2}{}
		\elemento{rf:5.1} d'imporre restrizioni, per esempio definire ora e giorno dell'attività da svolgere;
		\elemento{rf:5.2} impostare la priorità dell'impegno;
		\elemento{rf:5.3} scrivere una descrizione e un titolo;
		\elemento{rf:5.4} definire impegni di routine oppure impegni ripetuti su più giorni.
		\elemento{rf:5.5} l'utente avrà la possibilità d'impostare le ore di sonno giorno per giorno oppure applicare le stesse ore per più giorni;
		\elemento{rf:5.6} aggiungere il luogo dove si terrà l'impegno;
		\elemento{rf:5.7} impostare le autorizzazioni per eventuali utenti con cui verrà condiviso l'evento.
	\end{listaPersonale2}


	\elemento[IMPOSTAZIONE IMPEGNO]{rf:6} Una volta compilato il pop-up di aggiunta impegno, il sistema deve inserire l'impegno nel calendario (\ref{ob:3}) seguendo le restrizioni e priorità inserite dall'utente a tempo di compilazione dell'evento (\ref{rf:5}).

	\elemento[NOTIFICHE]{rf:7} Il sistema deve inviare delle notifiche per ciascun impegno (\ref{ob:4}), fornendo la possibilità di personalizzarle, infatti nel pop-up di compilazione dell'evento (\ref{rf:5}) deve essere possibile:
	\begin{listaPersonale2}{}
		\elemento{rf:7.1} impostare quando ricevere tale notifica
		\elemento{rf:7.2} definire il titolo della notifica: di default quest'ultimo sarà il titolo dell'evento.
	\end{listaPersonale2}

	\elemento[INTERAZIONE CON SISTEMI DI TERZE PARTI]{rf:8} Il sito deve permettere all'utente d'interagire con sistemi di terzi parti (\ref{ob:5}), ad esempio:

	\begin{listaPersonale2}{RF}
		\elemento[INTERAZIONE con GOOGLE CALENDAR]{rf:8.1} In quanto è un'applicazione molto diffusa tra gli utenti che utilizzano calendari, il sistema deve rendere possibile all'utente di poter interagire con eventi di Google Calendar. In particolare la piattaforma deve permettere di:
		\begin{listaPersonale3}{}
			\elemento{rf:8.1.1} importare ed esportare manualmente;
			\elemento{rf:8.1.2} importare ed esportare in automatico.
		\end{listaPersonale3}

		\elemento[INTERAZIONE CON UN SERVIZIO DI MAPPE]{rf:8.2} Il sistema darà la possibilità all'utente di aggiungere il luogo dove si svolgerà l'impegno.
	\end{listaPersonale2}

	\elemento[INFORMAZIONI sull'USO del TEMPO]{rf:9} Il sito deve presentare delle infografiche sull'uso tempo. L'utente, così, potrà visualizzare dei grafici esemplificativi su come viene speso il proprio tempo. (\ref{ob:12})

	\elemento[RIORGANIZZAZIONE DI ATTIVITA']{rf:10} Il sistema deve riorganizzare automaticamente il calendario in caso di ritardi (\ref{ob:3}). Sarà data la possibilità all'utente di notificare il sistema di ritardi e il sito dovrà ripianificare il calendario, sempre secondo le regole descritte nel \ref{rf:6}.

	\elemento[RESOCONTO GIORNATA]{rf:11} Il sistema, a fine giornata, deve presentare un resoconto (\ref{ob:7}), dove l'utente potrà comunicare le attività fatte e non, in modo tale da dare la possibilità al sistema di ricalcolare eventuali modifiche in base agli impegni non conclusi sempre secondo le modalità citate in \ref{rf:6}

	\elemento[FILTRO IMPEGNI]{rf:12} Il sito offrirà la possibilità di poter visualizzare gli impegni dato un specifico filtro(\ref{ob:10}) definito dall'utente secondo alcuni criteri:
	\begin{listaPersonale2}{}
		\elemento{rf:12.1} Titolo evento con corrispondenza totale o parziale.
		\elemento{rf:12.2} Data evento.
		\elemento{rf:12.3} Priorità evento.
		\elemento{rf:12.4} Persone incluse nell'evento.
	\end{listaPersonale2}

	\elemento[IMPOSTAZIONI PREDEFINITE DI UN CALENDARIO]{rf:13} Ogni calendario avrà una sezione dedicata dove l'utente potrà modificare le impostazioni predefinite del calendario. Queste impostazioni saranno usate per precompilare i campi dell'evento, che si sta aggiungendo. L'utente, durante la fase di aggiunta dell'evento, potrà modificare i campi precompilati. La sezione delle impostazioni di un calendario conterrà la possibilità di definire i valori di default per i seguenti requisiti funzionali: \ref{rf:5.1}, \ref{rf:5.2}, \ref{rf:5.3}(come prefisso o suffisso), \ref{rf:5.5}, \ref{rf:5.6}, \ref{rf:5.7}, \ref{rf:7.1}, \ref{rf:7.2} e inoltre questa sezione conterrà le impostazioni per:
	\begin{listaPersonale2}{}
		\elemento{rf:13.1} modificare il nome del calendario;
		\elemento{rf:13.2} modificare il colore del calendario.
	\end{listaPersonale2}

	\elemento[IMPOSTAZIONI ACCOUNT]{rf:14} Il sito permetterà all'utente di modificare delle impostazioni riguardo al proprio account; i campi che si potranno modificare saranno:
	\begin{listaPersonale2}{}
		\elemento{rf:14.1} modifica password;
		\elemento{rf:14.2} modifica username;
		\elemento{rf:14.3} impostare preferenze e interazioni con servizi di terze parti RF 8.1 e RF 8.2.
	\end{listaPersonale2}
\end{listaPersonale}