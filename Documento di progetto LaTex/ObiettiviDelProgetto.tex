\section{Obiettivi del progetto}
Il progetto ha come obiettivo la realizzazione di un \nome automatizzato.\\
Il sito si pone come un sistema automatizzato di programmazione del tempo in base agli impegni che si inseriscono, avendo la possibilità d'imporre delle preferenze per la disposizione delle attività da svolgere.\\
La funzione principale del \nome è la facilitazione del processo di programmazione del proprio tempo.

\vspace{0.5cm}

Nello specifico questa applicazione permette di:
\begin{itemize}
    \item compilare tramite un pop-up un impegno da aggiungere al proprio calendario; 
    \item poter applicare delle restrizioni riguardo a quando l'impegno può esser posto. Queste
    restrizioni si potranno imporre quando si compila il pop-up;
    \item impostare la priorità dell'impegno. La priorità potrà essere decisa quando si compilerà il pop-up in base all'importanza dell'attività da svolgere;
    \item aggiungere una descrizione dell'impegno nella compilazione del pop-up oltre al titolo dell'attività.
    \item avere la possibilità di gestire anche routine e impegni ripetuti; infatti nel pop-up sarà possibile anche 
    definire per quanti e per quali giorni si ha un'attività;
    \item una volta creato un impegno viene automaticamente impostata una data a meno di restrizioni. Precisamente, grazie alle restrizioni e alle priorità decise per ciascuno
    impegno, il sistema software sarà in grado di decidere quando porre tale attività;
    \item impostare preferenze su quanto e quando dormire. Sarà possibile modificare le ore di sonno giorno per giorno oppure
    applicare le stesse ore per più giorni;
    \item notifiche per ciascuno impegno;
    \item ogni evento ha la possibilità d'impostare notifiche personalizzate nel pop-up di aggiunta dell'impegno. Si potrà impostare, ad esempio, quando ricevere
    la notifica;
    \item importare eventi da Google Calendar.
    \item infografiche sull'uso del tempo. L'utente avrà la possibilità di visualizzare dei grafici esemplificativi di come il proprio
    tempo viene speso;
    \item riorganizzare automaticamente le altre attività in caso di ritardi. L'utente avrà la possibilità di notificare il sistema del ritardo e 
    a quel punto il sistema riorganizzerà il calendario sempre in base delle restrizioni e priorità poste per ciascuna attività.
    \item ottenere un resoconto a fine giornata per poter comunicare le attività fatte e non, in modo tale da dare la possibilità al sistema di ricalcolare eventuali modifiche in base agli impegni non conclusi
\end{itemize}

\todo{Finire Obiettivi}
