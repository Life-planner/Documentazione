\section{Obiettivi del progetto}
\label{sec:ObiettiviProgetto}
Il progetto ha come obiettivo la realizzazione di un calendario automatizzato.\\
Il sito si pone come un sistema automatizzato di programmazione del tempo in base agli impegni che si inseriscono, avendo la possibilità d'imporre delle preferenze per la disposizione delle attività da svolgere.\\
La funzione principale della piattaforma è la facilitazione del processo di programmazione del proprio tempo.

\vspace{0.5cm}

Nello specifico questa applicazione permette di:
\begin{listaPersonale}{OB}
      \elemento {ob:1} registrarsi al sito e autenticarsi con servizi di terzi parti, in modo tale da semplificare la registrazione e futuri accessi da parte dell'utente e di permettere agli utenti non registrati di accedere ad alcune funzionalità della piattaforma;
      \elemento {ob:2} aggiungere eventi al calendario e poterli gestire in modo tale da personalizzarli secondo le proprie preferenze;
      \elemento {ob:3} formattare automaticamente il calendario in base agli impegni e ai ritardi così da ottenere delle giornate sempre organizzate;
      \elemento {ob:4} avere notifiche per ciascun impegno avendo la possibilità di personalizzarle in base alle proprie preferenze;
      \elemento {ob:5} interagire con servizi di terze parti per aumentare le funzionalità offerte dalla piattaforma ed essere adattabile a dispositivi di dimensione variabile;
      \elemento {ob:6} avere accesso a funzioni premium e gestirne il pagamento in modo da offrire più funzionalità e rendere il servizio offerto economicamente sostenibile;
      \elemento {ob:7} ottenere un resoconto a fine giornata delle attività;
      \elemento {ob:8} gestire più calendari e interagire con calendari di altri utenti per poter organizzare attività in comune;
      \elemento {ob:9} interagire con l'applicativo in modo sicuro in modo tale che eventuali attacchi di rete non permettano di carpire alcuna informazione privata dell'utente; inoltre il sito deve garantire l'adempimento delle leggi sulla privacy presenti nel GDPR dell'Unione europea;
      \elemento {ob:10} filtrare gli impegni in base a dei criteri di ricerca;
      \elemento {ob:11} avere una web app in modo tale da estendere le funzionalità e portabilità del sito;
      \elemento {ob:12} ottenere informazioni sull'uso del tempo così da tenere gli utenti aggiornati sulle giornate;
      \elemento {ob:13} avere un qualsiasi numero di utenti attivi nel sito contemporaneamente;
      \elemento {ob:14} avere un sito affidabile, prestante, compatibile con diversi browser e che sia in italiano;
\end{listaPersonale}
