\section{Obiettivi del progetto}
\label{sec:ObiettiviProgetto}
Il progetto ha come obiettivo la realizzazione di un calendario automatizzato.\\
Il sito si pone come un sistema automatizzato di programmazione del tempo in base agli impegni che si inseriscono, avendo la possibilità d'imporre delle preferenze per la disposizione delle attività da svolgere.\\
La funzione principale di \nome è la facilitazione del processo di programmazione del proprio tempo.

\vspace{0.5cm}

Nello specifico questa applicazione permette di:
\begin {listaPersonale} {OB}
      \elemento {ob.one} registrarsi al sito;
      \elemento {ob.two} autenticarsi con servizi di terzi parti;
      \elemento {ob.three} aggiungere un impegno al proprio calendario;
      \elemento {ob.four} gestire gli impegni;
      \elemento {ob.five} ottenere un calendario automaticamente formato;
      \elemento {ob.six} impostare preferenze su quanto e quando dormire;
      \elemento {ob.seven} notifiche per ciascun impegno;
      \elemento {ob.eight} interagire con Google Maps;
      \elemento {ob.nine} effettuare pagamenti;
      \elemento {ob.ten} infografiche sull'uso del tempo;
      \elemento {ob.eleven} riorganizzare automaticamente le altre attività in caso di ritardi;
      \elemento {ob.twelwe} ottenere un resoconto a fine giornata delle attività;
      \elemento {ob.thirteen} interagire con i calendari di altre persone;
      \elemento {ob.fourteen} gestire più calendari;
      \elemento {ob.fifteen} interagire con l'applicativo in modo sicuro.
      \elemento {ob.sixteen} filtrare gli impegni;
      \elemento {ob.seventeen} interagire con Google Calendar;
      \elemento {ob.eighteen} avere una web app;
      \elemento {ob.nineteen} avere accesso a funzioni premium.
\end{listaPersonale}
