\section{Requisiti Front-End}
\label{sec:RequisitiFrontEnd}

Nel presente capitolo vengono riportati alcuni mockup relativi alle schermate dell’applicazione per Web da realizzare. Queste schermate hanno l’obiettivo di rappresentare come l’applicazione si dovrà presentare all’utente finale, il front end (FE), nel caso dei seguenti requisiti funzionali descritti precedentemente: 
\begin{itemize}
    \item registrazione e login nella piattaforma (RF1.1);
    \item interazione con calendari di altre persone (RF4.1);
    \item compilazione/modifica evento (RF5);
    \item impostazione impegno (RF6);
    \item interazione con un servizio di mappe (RF8.2);
    \item riorganizzazione di attività (RF10);
    \item informazioni sull’uso del tempo (RF9);
\end{itemize}

\begin{listaPersonale}{FE}
    \elemento[LOGIN NELLA PIATTAFORMA]{fe:1} In questa schermata, l’utente già registrato mediante email e password, deve inserire le credenziali da lui scelte per accedere al sito. Invece, coloro che hanno fatto accesso grazie all’utilizzo di servizi di terze parti, devono proseguire la loro autenticazione premendo il tasto corrispondente al servizio esterno da loro utilizzato per registrarsi (RF1.1).
    Se l’utente non si è ancora registrato sul sito deve schiacciare “Registrati” oppure proseguire l’autenticazione con uno dei servizi di terze parti offerti, ovvero Google ed Apple.
    \elemento [SCHERMATA PRINCIPALE] {fe:2} Questa schermata è la schermata principale della piattaforma PlanIt. L’utente, dopo aver fatto l’accesso al sito oppure proseguito con la modalità demo, giungerà a questa pagina, dove potrà guardare gli eventi posti nella settimana. L’utente ha la possibilità, dal menù al di sopra del calendario, di spostarsi di data sia in mesi che in settimane e accedere agli altri calendari personali e a quelli condivisi (RF4.1). Infine è presente anche il bottone "+” con cui si accede al pop-up di compilazione/aggiunta evento(RF5).
    \begin{listaPersonale2}{FE}
        \elemento[SCHERMATA PRINCIPALE-CALENDARI]{fe:2:1} Grazie alla sezione “Calendari” l’utente aprirà una sezione da dove può gestire i propri calendari, divisi per tematiche, e i calendari condivisi di altre persone (RF4.1). I calendari hanno, alla sinistra del loro nome, un colore per poterli differenziare velocemente. Inoltre, i calendari condivisi sono indicati con una piccola immagine stereotipata di persona, alla destra del loro nome.
    \end{listaPersonale2}
    \elemento[DASHBOARD] {FE:3} L’utente, nella dashboard, può osservare le informazioni principali riguardo al proprio calendario, le sezioni presenti sono: attività svolte questa settimana, situazione scadenza attività e attività svolte oggi con grafico attività svolte.
    \begin{listaPersonale2}{FE}
        \elemento[ATTIVITA' SVOLTE QUESTA SETTIMANA] {fe:3:1} Nella sezione “Attività svolte questa settimana” viene mostrato un grafico a barre delle varie attività svolte per ogni giorno della settimana (RF9), l’altezza delle barre corrisponde al quantitativo di ore dedicate a quell’attività.
        \elemento[SITUAZIONE SCADENZA ATTIVITA'] {fe:3:2} Nella sezione “Situazione scadenze attività” è presente in basso una heatmap riguardo al tempo che deve essere dedicato ogni giorno per rispettare le varie deadline(RF9). La legenda sopra la heatmap mostra che più è tendente il colore al rosso, più ore devono essere impiegate nello specifico periodo di tempo.
        \elemento [ATTIVITA’ SVOLTE OGGI e GRAFICO ATTIVITA' SVOLTE] {fe:3:3} Nella sezione “Attività svolte oggi” è presente una lista delle attività della giornate, da cui si possono ottenere anche le varie sottoattività. 
        La sezione “Attività svolte oggi” è sincronizzata con il grafico a torta presente nel “Grafico attività svolte” (RF9). Il grafico presenta le attività presenti in “Attività svolte oggi” con una dimensione proporzionata al tempo da spendere.         
    \end{listaPersonale2}
    \elemento [SCHERMATA ATTIVITA'] {fe:4} Nella schermata attività è presente una tabella delle attività della giornata con le varie informazioni, ovvero: titolo, descrizione, categoria, priorità, durata (RF5), posizione (RF8.2). Inoltre ci sono dei tasti con cui si può rimandare e ritardare l’attività (RF10) e, infine, un tasto per indicare di averla completata. A fine giornata di default tutti gli impegni sono posti come completati; per modificare tale opzione deve intervenire l’utente.
    \elemento [SCHERMATA EVENTI] {fe:5} Nella schermata eventi sono presenti tutti i comandi che riguardano la compilazione, modifica degli eventi (RF5) , creazione di un calendario (RF 2.2.2), creazione di raggruppamento di attività ed eliminazione di un evento (RF5).
    Nella sezione sottostante, è presente una lista dei vari calendari e raggruppamenti, che selezionati una alla volta aprono i rispettivi form di modifica e creazione, dove sono presenti tutti i campi citati in RF5 per l’aggiunta e modifica di impegni, RF13 per modifica e aggiunta di calendario. 
    \begin{listaPersonale2}{FE}
        \elemento[SCHERMATA CREA/MODIFICA EVENTO] {fe:5:1}
        \elemento[SCHERMATA CREA/MODIFICA RAGGRUPPAMENTO] {fe:5:2}
        \elemento[SCHERMATA CREA/MODIFICA CALENDARIO] {fe:5:3}
    \end{listaPersonale2}
    

    


\end{listaPersonale}


(Inoltre i cookie saranno usati per mantenere l'accesso così da evitare di fare il login ogni volta che si vuole utilizzare la piattaforma.)