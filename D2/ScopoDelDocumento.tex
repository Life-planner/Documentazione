\section*{Scopo del documento}
\addcontentsline{toc}{section}{Scopo del documento}
Lo scopo di questo documento è presentare i requisiti funzionali, i requisiti non funzionali, l'analisi del contesto e l'analisi dei componenti del progetto \nome ideato da Gabriele Lacchin, Denis Lucietto ed Emanuele Zini.\\
Il seguente documento presenta:
\begin{itemize}
    \item \hyperref[secD2:RequisitiFunzionali]{i requisiti funzionali};
    \item \hyperref[secD2:RequisitiNonFunzionali]{i requisti non funzionali};
    \item \hyperref[secD2:AnalisiDeiComponenti]{l'analisi dei componenti};
    \item \hyperref[secD2:AnalisiDelContesto]{l'analisi del contesto}.
\end{itemize}


Il presente documento riporta la specifica dei requisiti di sistema del progetto PlanIt usando diagrammi in Unified Modeling Language (UML) e tabelle strutturate. Nel precedente documento sono stati definiti gli obiettivi del progetto (PERCHE') e i requisiti (COSA) usando solo il linguaggio naturale.

Ora i requisiti vengono specificati usando sia il linguaggio naturale sia linguaggi più formali e strutturati, UML per la descrizione dei requisiti funzionali e tabelle strutturate per la descrizione dei requisiti non funzionali. Dopo, per la descrizione dei diagrammi in UML, oltre che il linguaggio naturale, si sono usati anche altri tipo di diagrammi, come lo Swimlanes Diagram e il Sequence Diagram. Inoltre, tenendo conto di tali requisiti, viene presentato il design del sistema con l'utilizzo di diagrammi di contesto e dei componenti, i quali sono accompagnati da dettagliate descrizioni in linguaggio naturale.
