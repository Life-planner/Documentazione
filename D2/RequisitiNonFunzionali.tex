\section{Requisiti non funzionali}
\label{secD2:RequisitiNonFunzionali}

\begin{listaPersonale}{RNF}

    \elemento[Privacy (\prettyref{D1-rnf:Privacy})]{rnfd2:Privacy}
    \begin{tabular}{|p{0.3\linewidth}|p{0.3\linewidth}|p{0.3\linewidth}|}
        \hline
        \rowcolor{viola} \textbf{Proprietà}                                                                                   &
        \textbf{Descrizione}                                                                                                  &
        \textbf{Misura}                                                                                                                              \\
        \hline
        Dati personali                                                                                                        &
        I dati personali dell'utente sono necessari per il
        legittimo interesse di utilizzo del sito, il consenso
        sarà richiesto solo per alcune funzionalità. In caso di
        mancata concessione l'utente non potrà avere accesso a tali
        funzionalità (\prettyref{D1-rnf:TrattamentoLecitoPrivacy})                                                            &
        In caso di rifiuto di condivisione di dati necessari per il funzionamento di una funzionalità quest'ultima non sarà utilizzabile dall'utente \\
        \hline
        Trattamento dei dati                                                                                                  &
        Trattamento dei dati è rappresentato da una pagina chiamata
        "termini e condizioni" che deve essere accettata dall'utente per
        poter creare un account. Tale pagina riporta e descrive le
        finalità e le modalità di trattamento dei dati quali: categoria
        di dati raccolti, finalità del
        trattamento e periodo di conservazione(\prettyref{D1-rnf:InformativaFinalitaPrivacy})                                 &
        In fase di registrazione l'utente dovrà accettare i termini
        e le condizioni presenti nell'apposita pagina per procedere
        con la creazione dell'account                                                                                                                \\
        \hline
        Accessibilità                                                                                                         &
        Accessibilità dei contenuti presenti nella pagina di "termini e condizioni" (\prettyref{D1-rnf:PresentazionePrivacy}) &
        La pagina di "termini e condizioni" deve essere scritta in
        modo che sia facilmente comprensibile a qualsiasi utente,
        deve presentare un testo con un font di facile lettura e
        avere uno sfondo e colore di testo che non impediscano in
        alcun modo la lettura del contenuto                                                                                                          \\
        \hline
        Diritto di accesso ai dati da parte dell'utente                                                                       &
        Per diritto di accesso ai dati da parte dell'utente si
        intende la possibilità da parte dell'utente di richiedere
        tutti i dati salvati dalla piattaforma relativi al proprio
        account (\prettyref{D1-rnf:DirittoAccessoDatiPrivacy})                                                                &
        Dalla pagina d'impostazioni dell'account l'utente deve avere
        la possibilità di fare richiesta di ricevere i propri dati
        salvati dalla piattaforma e riceverli in un intervallo di
        tempo minore di cinque giorni                                                                                                                \\
        \hline
        Diritto di rettifica                                                                                                  &
        Per diritto di rettifica si intende la possibilità da parte
        dell'utente di richiedere la rettifica dei dati presenti nel sistema.
        (\prettyref{D1-rnf:DirittoRettificaPrivacy})                                                                          &
        Tramite una richiesta via email e dopo aver effettuato una
        verifica che l'utente che ha richiesto la verifica è l'effettivo proprietario dell'account l'utente potrà
        richiedere la rettifica dei dati inesatti                                                                                                    \\
        \hline
        Diritto all'oblio / cancellazione                                                                                     &
        Per diritto all'oblio/cancellazione si intende la possibilità
        dell'utente di eliminare il
        proprio account e i dati connessi a esso (\prettyref{D1-rnf:DirittoOblioPrivacy})                                     &
        Nella pagina di gestione dell'account di un utente deve essere
        presente la voce per poter eliminare il proprio account e
        tutti i dati connessi a esso                                                                                                                 \\
        \hline
    \end{tabular}
    \newpage
    \begin{tabular}{|p{0.3\linewidth}|p{0.3\linewidth}|p{0.3\linewidth}|}
        \hline
        Diritto di opposizione                                        &
        Rappresenta la possibilità dell'utente di opporsi al trattamento dei suoi
        dati personali (\prettyref{D1-rnf:DirittoOpposizionePrivacy}) &
        Tramite una richiesta via email e dopo aver effettuato una
        verifica che l'utente che ha richiesto l'esercizio del suo diritto di opposizione è l'effettivo proprietario dell'account potrà richiedere che i suoi dati
        non vengano trattati ulteriormente                                   \\
        \hline
        Esercizio dei propri diritti                                  &
        Rappresenta il modo in cui l'utente può esercitare i propri diritti
        (\prettyref{D1-rnf:EsercizioDeiDirittiPrivacy})               &
        Tramite una richiesta via email e dopo aver effettuato una
        verifica che l'utente che vuole esercitare i propri diritti è l'effettivo proprietario dell'account potrà esercitare
        i propri diritti senza nessun altra necessità                        \\
        \hline
        Informazioni sulla violazione dei sistemi                     &
        Rappresenta il diritto dell'utente di essere informato in caso ci
        siano violazioni riguardo ai propri dati in possesso della piattaforma
        (\prettyref{D1-rnf:ViolazioneSistemaPrivacy})                 &
        In caso di violazione dei dati personali degli utenti dovrà essere
        notificata tale violazione entro 72 ore dalla scoperta dell'accaduto \\
        \hline
    \end{tabular}

    \elemento[Sicurezza (\prettyref{D1-rnf:Sicurezza})]{rnfd2:Sicurezza}
    \begin{tabular}{|p{0.3\linewidth}|p{0.3\linewidth}|p{0.3\linewidth}|}
        \hline
        \rowcolor{viola} \textbf{Proprietà}                                                                                      &
        \textbf{Descrizione}                                                                                                     &
        \textbf{Misura}                                                                                                            \\
        \hline
        Password                                                                                                                 &
        Password decisa obbligatoriamente dagli utenti in fase di
        registrazione non mediante servizi di terze parti (\prettyref{D1-rnf:SicurezzaPassword})                                 &
        La password deve essere formata da almeno otto caratteri e
        deve contenere almeno un carattere per ognuno di questi tre elementi:
        lettera maiuscola, lettera minuscola, numero e simboli (!\@\#\$\%\^\&*)                                                    \\
        \hline
        Prevenzione attacchi di “sniffing” o “man in the middle”                                                                 &
        Modalità di trasmissione dei dati dell'utente (\prettyref{D1-rnf:AttacchiSicurezza})                                     &
        Utilizzo protocollo https                                                                                                  \\
        \hline
        Politiche sull'origine dei dati                                                                                          &
        L'accesso alle api dell'applicazione devono essere limitate alla stessa origine (\prettyref{D1-rnf:OrigineDatiSicurezza} &
        Le politiche CORS impostate sulle api dell'applicativo
        devono essere impostate su Same-Origin Policy                                                                              \\
        \hline
        Prevenzione attacchi “XSS”                                                                                               &
        Modalità di trattamento input inseriti dall'utente (\prettyref{D1-rnf:XSSSicurezza})                                     &
        Sanificazioni dei dati inseriti dall'utente                                                                                \\
        \hline
        Prevenzione attacchi DDOS                                                                                                &
        Modalità di protezione abilitata sui server di hosting del sistema (\prettyref{D1-rnf:DDOSSicurezza})                    &
        Il firewall dei server di hosting del sistema devono implementare
        un sistema che blocchi ogni attacchi di tipo DDOS                                                                          \\
        \hline
    \end{tabular}

    \elemento[Scalabilità (\prettyref{D1-rnf:Scalabilita})]{rnfd2:Scalabilita}
    \begin{tabular}{|p{0.3\linewidth}|p{0.3\linewidth}|p{0.3\linewidth}|}
        \hline
        \rowcolor{viola} \textbf{Proprietà}                                         &
        \textbf{Descrizione}                                                        &
        \textbf{Misura}                                                                                                    \\
        \hline
        Elaborazione con un numero crescente di utenti                              &
        Capacità del sistema di gestire un numero crescente di utenti in simultanea &
        Garantita fino a 5000 utenti in simultanea                                                                         \\
        \hline
        Memorizzazione dei dati con un numero alto di utenti                        &
        Capacità del sistema di gestire i dati generati da un alto numero di utenti &
        Garantita una scalabilità per il tempo di richiesta dei dati lineare del database in relazione al numero di utenti \\
        \hline
    \end{tabular}

    \newpage
    \elemento[Affidabilità (\prettyref{D1-rnf:Affidabilita})]{rnfd2:Affidabilita}
    \begin{tabular}{|p{0.3\linewidth}|p{0.3\linewidth}|p{0.3\linewidth}|}
        \hline
        \rowcolor{viola} \textbf{Proprietà}                                                         &
        \textbf{Descrizione}                                                                        &
        \textbf{Misura}                                                                               \\
        \hline
        Tempo medio di uptime                                                                       &
        Per tempo medio di uptime si intende il
        numero minimo di giorni in cui il sistema
        deve essere raggiungibile                                                                   &
        97\% che equivale a circa 354 giorni all'anno di uptime                                       \\
        \hline
        Conferma azione                                                                             &
        Per conferma  azione si intende il feedback
        dato dal sistema al compiersi di una azione
        da parte dell'utente (\prettyref{D1-rnf:ConfermaServerAffidabilita})                        &
        Il server deve dare un messaggio di conferma di
        modifica avvenuta quando l'utente effettua un azione                                          \\
        \hline
        Eccezioni                                                                                   &
        Per eccezioni si intendono tutti i vari errori che
        possono avvenire all'interno del sistema (\prettyref{D1-rnf:GestioneEccezioniAffidabilita}) &
        Il sistema deve gestire tutte queste eccezioni con
        errori appropriati e evitando il crash del sistema                                            \\
        \hline
        Resilienza                                                                                  &
        Per resilienza si intende la capacità del sistema di
        essere disponibile anche con il fallimento di
        parte dei componenti di cui è formato (\prettyref{D1-rnf:ResilienteFallimentoAffidabilita}) &
        In caso del fallimento di uno dei componenti della
        piattaforma esso non deve intaccare il funzionamento
        del resto dei componenti della piattaforma                                                    \\
        \hline
    \end{tabular}

    \elemento[Prestazioni (\prettyref{D1-rnf:Prestazioni})]{rnfd2:Prestazioni}
    \begin{tabular}{|p{0.3\linewidth}|p{0.3\linewidth}|p{0.3\linewidth}|}
        \hline
        \rowcolor{viola} \textbf{Proprietà}                           &
        \textbf{Descrizione}                                          &
        \textbf{Misura}                                                                   \\
        \hline
        Avvio applicazione                                            &
        Tempo massivo di avvio dell'applicazione da parte dell'utente &
        Quando l'utente apre l'applicativo con una connessione
        fast 3G (velocità: ~1.44 Mbps e latenza: ~562.5ms )
        il sito deve essere disponibile entro 3 secondi.                                  \\
        \hline
        Lighthouse score                                              &
        Misurazione del punteggio dell'applicazione tramite un
        sistema di benchmark chiamato Lighthouse comprensivo
        di tale categorie, ovvero:
        \begin{itemize}
            \item performance;
            \item accessibility;
            \item best practises;
            \item SEO;
        \end{itemize}                                         &
        Il sito deve ottenere un punteggio per ogvni categoria di almeno 85\% o superiore \\
        \hline
    \end{tabular}

    \newpage
    \elemento[Compatibilità (\prettyref{D1-rnf:Compatibilita})]{rnfd2:Compatibilita}
    \begin{tabular}{|p{0.3\linewidth}|p{0.3\linewidth}|p{0.3\linewidth}|}
        \hline
        \rowcolor{viola} \textbf{Proprietà} &
        \textbf{Descrizione}                &
        \textbf{Misura}                       \\
        \hline
        Compatibilità con Chrome            &
        Browser e versione a partire dalla
        quale la piattaforma è compatibile  &
        L'applicativo deve essere
        compatibile con la versione
        84 di Chrome e successive versioni.   \\
        \hline
        Compatibilità con Edge              &
        Browser e versione a partire dalla
        quale la piattaforma è compatibile  &
        L'applicativo deve essere
        compatibile con la versione
        84 di Edge e successive versioni.     \\
        \hline
        Compatibilità con Safari            &
        Browser e versione a partire dalla
        quale la piattaforma è compatibile  &
        L'applicativo deve essere
        compatibile con la versione
        14.1 di Safari e successive versioni. \\
        \hline
        Compatibilità con Firefox           &
        Browser e versione a partire dalla
        quale la piattaforma è compatibile  &
        L'applicativo deve essere
        compatibile con la versione
        63 di Firefox e successive versioni.  \\
        \hline
    \end{tabular}

    \elemento[Web App (\prettyref{D1-rnf:WebApp})]{rnfd2:WebApp}
    \begin{tabular}{|p{0.3\linewidth}|p{0.3\linewidth}|p{0.3\linewidth}|}
        \hline
        \rowcolor{viola} \textbf{Proprietà}                      &
        \textbf{Descrizione}                                     &
        \textbf{Misura}                                            \\
        \hline
        Web App                                                  &
        Descrive da che dispositivi è accessibile la piattaforma &
        L'applicativo deve essere utilizzabile sia da
        browser per la versione PC e sia tramite applicazione
        mobile derivata dalla web application.                     \\
        \hline
    \end{tabular}

    \newpage
    \elemento[Interoperabilità (\prettyref{D1-rnf:Interoperabilita})]{rnfd2:Interoperabilita}
    \begin{tabular}{|p{0.3\linewidth}|p{0.3\linewidth}|p{0.3\linewidth}|}
        \hline
        \rowcolor{viola} \textbf{Proprietà}                                                                         &
        \textbf{Descrizione}                                                                                        &
        \textbf{Misura}                                                                                               \\
        \hline
        Interazione con sistemi di mappe                                                                            &
        Per interazione con sistemi di mappe, si intende la
        possibilità la presenza di funzionalità aggiuntive,
        riguardo alla gestione della localizzazione, date
        sistemi esterni di mappe (\prettyref{D1-rnf:MappeInteroperabilita}). Nello specifico questi
        sistemi di mappe sono:
        \begin{itemize}
            \item Google Maps
            \item OpenStreetMap
        \end{itemize}                                                                                         &
        Quando l'applicativo manda una richiesta d'indirizzo
        al sistema di mappe, questo deve ritornare le sue
        coordinate corrette. Il sistema deve supportare sia
        Google Maps che OpenstreetMaps                                                                                \\
        \hline
        Interazione con sistemi di pagamento                                                                        &
        Per interazione con sistemi di pagamento, si intende
        la presenza di funzionalità aggiuntive, riguardo a
        gestione di pagamenti, date da sistemi di pagamento (\prettyref{D1-rnf:PagamentoInteroperabilita}).
        Nello specifico questi sistemi di pagamento sono:
        \begin{itemize}
            \item PayPal
            \item Stripe
        \end{itemize}                                                                                             &
        Quando l'applicativo manda una richiesta di pagamento,
        questa deve essere presa in carico dal sistema di
        pagamento (PayPal o Stripe), che alla fine della
        procedura manderà un esito alla nostra piattaforma.                                                           \\
        \hline
        Interazione con sistema di autenticazione                                                                   &
        Per interazione con sistema di autenticazione,
        si intende la presenza di funzionalità aggiuntive,
        riguardo alla gestione di autenticazione,
        data dal sistema esterno di autenticazione (\prettyref{D1-rnf:AutenticazioneInteroperabilita}).             &
        L'applicativo deve implementare Auth0 per la
        gestione di registrazione e autenticazione
        degli utenti della piattaforma                                                                                \\
        \hline
        Integrazione con sistema di calendari                                                                       &
        Per interazione con sistema di calendari,
        si intende la presenza di funzionalità aggiuntive,
        riguardo all'importazione di dati da sistemi terzi
        e esportazione di dati verso calendari di sistemi terzi (\prettyref{D1-rnf:GoogleCalendarInteroperabilita}) &
        L'applicativo deve poter importare dati
        presenti su google calendar (calendari ed eventi)
        una volta connesso alla piattaforma e dare la
        possibilità anche di esportare dati della piattaforma
        verso Google calendar (calendari ed eventi)                                                                   \\
        \hline
    \end{tabular}

    \elemento[Portabilità (\prettyref{D1-rnf:Portabilita})]{rnfd2:Portabilita}
    \begin{tabular}{|p{0.3\linewidth}|p{0.3\linewidth}|p{0.3\linewidth}|}
        \hline
        \rowcolor{viola} \textbf{Proprietà} &
        \textbf{Descrizione}                &
        \textbf{Misura}                       \\
        \hline
        Adattività                          &
        Il sistema deve adattarsi correttamente
        alle dimensioni delle diverse schermate
        dei vari dispositivi, sia mobili come
        tablet e smartphone e sia non mobili
        quali dispositivi desktop           &
        Il sistema deve essere adattivo,
        cioè deve rispettare le proporzioni
        dei vari componenti dell'applicazione
        web sui vari dispositivi,
        come smartphone, tablet e PC          \\
        \hline
    \end{tabular}

    \elemento[Lingua Sistema (\prettyref{D1-rnf:Lingua})]{rnfd2:Lingua}
    \begin{tabular}{|p{0.3\linewidth}|p{0.3\linewidth}|p{0.3\linewidth}|}
        \hline
        \rowcolor{viola} \textbf{Proprietà} &
        \textbf{Descrizione}                &
        \textbf{Misura}                       \\
        \hline
        Lingua                              &
        Lingue previste su tutte le
        schermate di cui è composto
        il software                         &
        Le schermate devono
        essere in lingua italiana             \\
        \hline
    \end{tabular}

    \elemento[Cookies (\prettyref{D1-rnf:Cookie})]{rnfd2:Cookie}
    \begin{tabular}{|p{0.3\linewidth}|p{0.3\linewidth}|p{0.3\linewidth}|}
        \hline
        \rowcolor{viola} \textbf{Proprietà} &
        \textbf{Descrizione}                &
        \textbf{Misura}                       \\
        \hline
        Cookie                              &
        I cookie salvati sull'applicativo
        per rendere l'esperienza utente
        più fluida                          &
        Presenza di cookie per mantenimento
        delle informazione dell'utente e
        della sezione dopo che l'utente
        ne acconsente la presenza             \\
        \hline
    \end{tabular}

\end{listaPersonale}