\section{Obiettivi del progetto}
\label{sec:ObiettiviProgetto}
Il progetto ha come obiettivo la realizzazione di un calendario automatizzato.\\
Il sito si pone come un sistema automatizzato di programmazione del tempo in base agli impegni che si inseriscono, avendo la possibilità d'imporre delle preferenze per la disposizione delle attività da svolgere.\\
La funzione principale della piattaforma è la facilitazione del processo di programmazione del proprio tempo.

\vspace{0.5cm}

Nello specifico questa applicazione permette di:
\begin{listaPersonale}{OB}
      \elemento {ob:Registrazione} registrarsi al sito e autenticarsi con servizi di terzi parti, e di permettere agli utenti non registrati di accedere ad alcune funzionalità della piattaforma; inoltre dovranno essere disponibili delle impostazioni per l'account e la possibilità di recuperare la password per gli utenti autenticati;
      \elemento {ob:AggiuntaModificaEvento} aggiungere eventi al calendario e poterli gestire in modo personalizzato;
      \elemento {ob:FormattazioneAutomaticaCalendario} formattare automaticamente il calendario in base agli impegni e ai ritardi;
      \elemento {ob:NotifichePersonalizzate} avere notifiche personalizzabili per ciascun impegno;
      \elemento {ob:ServiziTerzeParti} interagire con servizi di terze parti ed essere adattabile a dispositivi di dimensione variabile;
      \elemento {ob:AccountPremium} avere accesso a funzioni premium e gestirne il pagamento;
      \elemento {ob:Resoconto} ottenere un resoconto a fine giornata delle attività;
      \elemento {ob:CalendariMultipli} gestire più calendari e interagire con calendari di altri utenti;
      \elemento {ob:SicurezzaPrivacyCookie} interagire con l'applicativo in modo sicuro, inoltre il sito deve garantire l'adempimento delle leggi vigenti in materia di privacy e di utilizzo dei cookie;
      \elemento {ob:Filtro} filtrare gli impegni in base a dei criteri di ricerca;
      \elemento {ob:WebApp} avere una web app;
      \elemento {ob:InfoUsoDelTempo} ottenere informazioni sull'uso del tempo;
      \elemento {ob:Scalabilita} supportare un numero crescente di utenti;
      \elemento {ob:Caratteristiche} avere un sito affidabile, prestante, compatibile con diversi browser e che sia in italiano;
\end{listaPersonale}
