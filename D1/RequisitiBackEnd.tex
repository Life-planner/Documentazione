\section{Design Back-End}
\label{sec:RequisitiBackEnd}

Nel presente capitolo vengono riportati i sistemi esterni con cui l'applicazione dovrà interfacciarsi per poter funzionare e una loro descrizione.

I sistemi esterni con cui PlanIt si dovrà interfacciare sono:
\begin{enumerate}
    \item sistemi di mappe: Google Maps e OpenStreetMap;
    \item sistema di gestione del calendario: Google Calendar;
    \item sistemi di pagamento: PayPal e/o Stripe;
    \item sistemi di autenticazione: Auth0;
    \item sistemi per la protezione da attacchi: Cloudflare;
    \item sistema di archiviazione delle informazioni: MongoDB;
    \item sistema di controllo della privacy e cookie: Iubenda;
    \item sistema di posta elettronica, email.
\end{enumerate}

\subsection*{DESCRIZIONI}
\begin{enumerate}
    \item Le API dei servizi di mappe Google Maps e OpenstreetMap, daranno la possibilità all'utente d'indicare la posizione esatta di dove si terrà l'impegno(\ref{rf:ServizioMappe}); tale posizione potrà essere specificata quando si compila il pop-up di compilazione impegno (\ref{rf:CreazioneModificaEvento}), oppure in un successivo momento durante la modifica dell'evento stesso (\ref{rf:CreazioneModificaEvento}).

    \item L'integrazione con Google Calendar(\ref{rf:GoogleCalendar}) permetterà d'importare ed esportare eventi dalla sua piattaforma.\\
          Il processo potrà avvenire mediante:
          \begin{itemize}
              \item L'importazione e l'esportazione tramite file: \\
                    Si potranno importare ed esportare eventi da Google Calendar mediante l'utilizzo di un file.
              \item L'importazione e l'esportazione in automatico: \\
                    Mediante l'uso delle API fornite dalla piattaforma Google Calendar, sarà possibile importare ed esportare eventi direttamente da e verso un calendario già esistente sulla piattaforma.
          \end{itemize}

    \item L'integrazione con i sistemi di pagamento PayPal e Stripe gestiranno i pagamenti per la sottoscrizione a un account premium (\ref{rf:PagamentoUtentePremium}).

    \item Il sistema di autenticazione Auth0 gestirà la procedura di accesso e registrazione al sito (\ref{rf:AccessoRegistrazione}), permettendo accessi e registrazioni sicure (\ref{rnf:Sicurezza}) alla piattaforma. Mediante l'utilizzo di questo sistema di accesso esterno sarà anche possibile accedere e registrarsi direttamente sul sito tramite l'utilizzo di account di servizi terzi, tra cui: Google e altri eventuali servizi di terzi parti.

    \item Il servizio CloudFlare verrà utilizzato per diversi scopi tra cui:
          \begin{enumerate}
              \item protezione da attacchi DDOS (\ref{rnf:Sicurezza});
              \item CDN (Content Delivery Network) in modo tale da servire le pagine statiche o file statici del sito con poca latenza;
              \item DNS così da poter creare diversi sotto domini;
              \item Certificato SSL, così da rendere sicura la connessione tra l'utente e la piattaforma (\ref{rnf:Sicurezza}).
          \end{enumerate}

    \item Il sistema di archiviazione delle informazioni MongoDB dovrà gestire tutti i dati coinvolti nei processi della piattaforma, nello specifico: la memorizzazione del calendario, degli eventi, eventuali condivisioni, le informazioni di ogni utente per l'integrazione con i servizi di terze parti e le preferenze dell'utente.

    \item Il sistema di gestione della privacy e cookie Iubenda sarà utilizzato per generare una politica sulla privacy (\ref{rnf:Privacy}) che rispetterà i parametri di legge, e in caso di modifiche alle leggi in vigore informerà la piattaforma per adottare le modifiche necessarie per adempiere ai nuovi obblighi di legge. Inoltre Iubenda fornirà un' integrazione per accettare/rifiutare i cookie (\ref{rnf:Cookie}).

    \item Il sistema di posta elettronica, email sarà utilizzato per inviare email informative sul sito e email relative a comunicazioni per l'utente, ovvero:
          \begin{itemize}
              \item reset password (\ref{rf:RecuperoPassword});
              \item conferma creazione account(\ref{rf:AccessoRegistrazione});
              \item promemoria scadenza piano a pagamento(\ref{rf:UtentePremium}).
          \end{itemize}
\end{enumerate}

\subsection*{Servizi esterni}
\begin{center}
    \includegraphics[width=1\textwidth]{img/Servizi/Servizi esterni.drawio.png}
\end{center}