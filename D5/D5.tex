\documentclass{article}

\usepackage[utf8]{inputenc}
\usepackage[italian]{babel}
\usepackage[T1]{fontenc}

\usepackage[a4paper,top=2cm,bottom=2cm,left=3cm,right=3cm]{geometry}
\usepackage{hyperref, xr, amsmath, amssymb, graphicx, mathtools, xcolor, enumitem, verbatim, todonotes, csquotes, subcaption, fancyhdr, lastpage, cancel, xspace, float, caption, prettyref}
\usepackage[most]{tcolorbox}

\externaldocument[D1-]{../D1/out/ObiettiviDelProgetto}
\newrefformat{D1-ob}{\color{blue}D1 \ref{#1}\color{black}}

\externaldocument[D1-]{../D1/out/RequisitiFunzionali}
\newrefformat{D1-rf}{\color{blue}D1 \ref{#1}\color{black}}

\externaldocument[D1-]{../D1/out/RequisitiNonFunzionali}
\newrefformat{D1-rnf}{\color{blue}D1 \ref{#1}\color{black}}

\externaldocument[D1-]{../D1/out/RequisitiFrontEnd}
\newrefformat{D1-fe}{\color{blue}D1 \ref{#1}\color{black}}

\externaldocument[D1-]{../D1/out/RequisitiBackEnd}
\newrefformat{D1-be}{\color{blue}D1 \ref{#1}\color{black}}
\title{Report Finale}
\author{Gruppo T56}
\date{A.A. 2022-2023}

%\setlength\parindent{0pt}

\setcounter{tocdepth}{5}
\setcounter{secnumdepth}{5}

\lhead{\textbf{Document: }Report Finale}
\rhead{\textbf{Revision: }0.1}
\cfoot{\thepage /\pageref{LastPage}}

\hypersetup{
    colorlinks=true,
    linkcolor=blue,
    urlcolor=blue,
    pdftitle={Report Finale - Gruppo T56},
}

\pagestyle{fancy}
\newcommand{\nome}[0]{Life Planner\xspace}

\begin{document}

\begin{titlepage}
    \begin{figure}[!htb]
        \minipage{0.5\textwidth}
        \includegraphics[width=0.7\textwidth]{img/logo_unitn.png}
        \endminipage
        \hfill
        \minipage{0.5\textwidth}
        \begin{flushright}
            \Large
            Dipartimento d'Ingegneria e Scienze dell'informazione
        \end{flushright}
        \endminipage
        \hfill
    \end{figure}

    \vspace{6cm}

    \large
    \textbf{Progetto:}
    \begin{center}
        \Huge
        \color{blue}
        \textbf{Life planner}
    \end{center}

    \vspace{1cm}

    \textbf{Titolo del documento:}
    \begin{center}
        \huge
        \color{blue}
        \textbf{Documento di progetto}\\
        \textbf{(Obiettivi e Requisiti)}
    \end{center}


    
\end{titlepage}
\pagebreak

\tableofcontents
\pagebreak

\section{Organizzazione del lavoro}

L'organizzazione del gruppo è stata fatta soprattutto per competenza. Infatti sia Denis Lucietto che Emanuele Zini avevano avuto esperienza di progettazione di siti web, dunque, fin da subito si era deciso che sarebbero stati coloro che avrebbero sviluppato il sito. Inoltre, visto le proprie preferenze e maggiori esperienze, il FrontEnd, invece da Emanuele Zini e di conseguenza il BackEnd è stato fatto da Denis Lucietto. Invece, Gabriele Lacchin si è concentrato, soprattutto, sulla scrittura dei documenti di progetto che ha sovrinteso, andando soprattutto a fare la stesura su Latex assieme a Denis vist la maggiore esperienza. E' stato scelto di usare Latex, per formulare i documenti di progetto, perché ritenuto più elegante, efficace, presentabile, versatile e personalizzabile per quello che è stato il progetto e il corso. Infatti, grazie a Latex, ci è stato possibile formulare un complicato, ma altrettanto efficace, sistema di linking sia all'interno del documento stesso, sia tra i vari documenti. Inoltre, grazie a Latex, ci è stato possibile anche formulare delle liste di elementi più comode per la presentazione si quello che dovevamo fare. Infine, grazie a Latex ci è stato possibile mantenere anche la parte di "Documentazione" su GitHub, in modo tale che tutto il progetto (documenti e immagini) oltre alla parte di codice fosse reperibile e modificabile usando GitHub. \\
Inoltre al seguente \href{https://www.notion.so/00afd2533fec4cae9120d119bfb7c9a1?v=332b02dda53744719f961bed09df9116}{link} si puo trovare una tabella con tutte le ore fatte da ciascun membro più nello specifico. Infatti, abbiamo deciso di creare questa dashboard, in cui ciascuno, ogni volta che lavorava, segnava ciò che ha fatto e il monte ore.\\
Ad ogni modo, c'è stata molta collaborazione in tutto il processo di costruzione del progetto. Infatti ogni settimana, venivano programmati dei meeting, di solito di martedì e giovedì per poter parlare di come stava andando il tutto, scambiarsi idee, dubbi e organizzare il lavoro da fare. Si può dire, per questo motivo, che tutti noi 3 siamo coscienti di come sono state fatte le cose, come sono andate e perché si è fatto in un certo modo; lo scambio di opinioni e idee è stato continuo anche con call su discord, oppure tramite messaggi.
\newpage
\section{Ruoli e attività}

\begin{center}
    \begin{tabular}{|c|p{0.2\linewidth}|p{0.3\linewidth}|}
        \hline
        Componente del team & Ruolo                                       & Principali attività                                                                                                                                                                                                                                                                                                                                                                                                                                                                                                                                                  \\
        \hline
        Gabriele Lacchin    & Analista / Progettista                      & Il ruolo principale è stato quello di analisi e progettista. Infatti, il suo compito è stato quello, soprattutto, di scrittura  e stesura della documentazione su Latex per ciascuno dei deliverables e la formulazione dei vari diagrammi: lavoro che ha dovuto fare per tutti i deliverables e che gli ha occupato più tempo nel D2 e D4. Nel D4, oltre alla scrittura del documento di progetto, ha contribuito alla fase di testing, fatta assieme a Denis.                                                                                                      \\
        \hline
        Denis Lucietto      & Analista / Progettista / Sviluppatore       & Nei primi documenti il suo ruolo è stato quello di analista e progettista, infatti ha lavorato molto assieme a Gabriele, soprattutto nel primo documento e buona parte del secondo. Dalla scrittura del D3, di cui ha fatto l'OCL e ha aiutato Emaunele nel diagramma delle classi, il suo ruolo si è sempre più delineato come sviluppatore; infatti è colui che ha implementato tutte le API, documentazione di queste con Swagger e una parte del testing. Per questo motivo, il d4 è stata la parte di progetto che gli ha occupato anche il più alto monte ore. \\
        \hline
        Emanuele Zini       & Project leader / Sviluppatore / Progettista & Il suo ruolo è stato quello, soprattutto, di progettare il sito assieme agli altri componenti del gruppo, ma concentrandosi di più sul FrontEnd. Infatti nel D1 è colui che ha fatto il mockup del sito e nel D4 ha sviluppato tutta la parte di FrontEnd. Negli altri documenti ha contribuito alla revisione, alla quasi completa realizzaione del diagrammi delle classe e assieme a Gabriele per quello delle componenti e stesura dei requisiti non funzionali nel D2.                                                                                          \\
        \hline
    \end{tabular}
\end{center}

\section{Carico e distribuzione del lavoro}

Scriviamo di seguito il motivo di eventuali squilibri e come è stato diviso il lavoro.\\







Nel D1 Gabriele e Denis hanno lavorato assieme alla scrittura e stesure degli obiettivi, impostazioni del documento requisiti funzionali e non funzionali. Emanuele ha fatto il mockup del FrontEnd, il quale dopo è stato messo nel deliverable da Gabriele, accompagnato dalla descrizione. Ad ogni modo, il lavoro nel D1 è stato abbastanza equilibrato. \\







Nel D2, Gabriele e Denis hanno continuato a lavorare assieme nella formulazione degli Use Case, Swimlanes Diagram, Sequence Diagram, descrizioni.
Di seguito, Gabriele ed Emanuele si sono concentrati sul diagramma di componenti, il quale ha occupato molto tempo. Infatti, è stato più volte modificato.Invece Denis si è concentrato soprattutto sul fare il diagramma di contesto di cui ha fatto, in buona parte, la descrizione. Gabriele ha molte più ore degli altri, perché, dopo aver fatto il diagramma di componenti assieme ad Emanuele, ha fatto descrizione di quasi tutte le componenti e interfacce. Emanuele, in questo documento, ha anche fatto la descrizione dei requisiti non funzionali.\\







Nel D3, il lavoro è stato diviso meglio. Infatti, Emanuele ha fatto il diagramma delle classi, Gabriele ha fatto la stesura del documento di progetto scrivendo le varie descrizioni e chiarimenti necessari mentre Denis ha fatto l'OCL e riportato su LaTex il tutto. Per questo motivo, in questo documento c'è stato abbastanza equilibrio nel monte ore.\\







Nel D4, Denis ha fatto le APIs e la relativa documentazione su per swagger, poi, assieme a Gabriele, ha fatto il testing delle API. Gabriele, invece, ha fatto i diagrammi e la stesura del documento di progetto D4. Infine, Emanuele ha fatto lo sviluppo del FrontEnd, in quanto è stato anche colui che ha progettato il mockup nel D1 e quindi aveva già in mente come farlo.\\







Sottolineiamo, che la divisione di lavori sopra descritta non è molto precisa, in quanto gli scambi di idee, revisioni reciproche e modifiche sono state continue e, quindi, in molte cose tutti hanno dato la loro parte. Infatti, la divisione di lavori, che è stata definita, si basa soprattutto su chi abbia impiegato più tempo a fare un specifico lavoro e non sono segnalate, ovviamente, eventuali correzioni, modifiche fatti dagli altri componenti al lavoro fatto.


\begin{center}
    \begin{tblr}{colspec={lccccc|c},process=\funcSum}
                         & D1     & D2     & D3     & D4     & D5     & TOT    \\
        Gabriele Lacchin & 51     & 81     & 22     & 52     & 0      & rowsum \\
        Denis Lucietto   & 40     & 66     & 27     & 73     & 1      & rowsum \\
        Emanuele Zini    & 49     & 34     & 31     & 0      & 0      & rowsum \\
        \hline
        Total            & colsum & colsum & colsum & colsum & colsum          \\
    \end{tblr}
\end{center}



\section{Criticità}


Le criticità che sono state individuate riguardano soprattutto l'organizzazione del lavoro del D1 e D2. Infatti, inizialmente, Gabriele e Denis hanno lavorato assieme alla stesura e scrittura del D1 e buona parte del D2. Ma questo sistema di lavoro, dopo un po', non lo abbiamo trovato più efficace e abbiamo deciso, più che altro, di dividerci il lavoro in modo netto e quello che andavamo a fare alla fine era una revisione reciproca del lavoro fatto. In questo modo, soprattutto nel D3 e d4, il lavoro è stato più efficace.

\section{Autovalutazione}

Vogliamo precisare che, nel complesso, il lavoro di gruppo svolto, è stato ritenuto da tutti efficace, buono e che siamo molto soddisfatti del progetto ottenuto. Pensiamo di aver fatto al meglio delle nostre capacità e di averci speso davvero molte energie e tempo tutti quanti. La collaborazione è stata alla base del gruppo insieme con gli scambi continui: non mentiamo nel dire che praticamente tutti i giorni parlavamo di come stava andando la nostra parte di lavoro, cosa ci mancava, per quando avremmo finito, ecc... \\
Visto che Denis e Gabriele sono sempre stati, praticamente, presenti e reperibili per lavorare e hanno quasi sempre, per tutto l'arco del semestre, continuato a lavorare sul progetto, abbiamo deciso che meritassero lo stesso voto, dato anche quasi lo stesso monte ore. \\
Invece, Emanuele, in quanto ha lavorato soprattutto la parte di FrontEnd, contribuendo solo in parte alle altre parti di progetto e, per questo motivo, ha meno ore di Gabriele e Denis, meriti un voto di differenza con Gabriele e Denis. Abbiamo deciso solo un voto di differenza, in quanto il lavoro fatto da Emanuele, anche se di meno rispetto a Gabriele e Denis, è sempre stato efficace e ad ogni modo Emanuele è sempre stato disponibile per eventuali chiarimenti e scambi di idee.

\begin{center}
    \begin{tabular}{|c|c|}
        \hline
                         & Voto \\
        \hline
        Gabriele Lacchin & 30   \\
        \hline
        Denis Lucietto   & 30   \\
        \hline
        Emanuele Zini    & 29   \\
        \hline
    \end{tabular}
\end{center}

\end{document}
