\documentclass{article}

\usepackage[utf8]{inputenc}
\usepackage[italian]{babel}
\usepackage[T1]{fontenc}

\usepackage[a4paper,top=2cm,bottom=2cm,left=3cm,right=3cm]{geometry}
\usepackage{hyperref, xr, amsmath, amssymb, graphicx, mathtools, xcolor, enumitem, verbatim, todonotes, csquotes, subcaption, fancyhdr, lastpage, cancel, xspace, float, caption, prettyref}
\usepackage[most]{tcolorbox}

\externaldocument[D1-]{../D1/out/ObiettiviDelProgetto}
\newrefformat{D1-ob}{\color{blue}D1 \ref{#1}\color{black}}

\externaldocument[D1-]{../D1/out/RequisitiFunzionali}
\newrefformat{D1-rf}{\color{blue}D1 \ref{#1}\color{black}}

\externaldocument[D1-]{../D1/out/RequisitiNonFunzionali}
\newrefformat{D1-rnf}{\color{blue}D1 \ref{#1}\color{black}}

\externaldocument[D1-]{../D1/out/RequisitiFrontEnd}
\newrefformat{D1-fe}{\color{blue}D1 \ref{#1}\color{black}}

\externaldocument[D1-]{../D1/out/RequisitiBackEnd}
\newrefformat{D1-be}{\color{blue}D1 \ref{#1}\color{black}}
\title{Report Finale}
\author{Gruppo T56}
\date{A.A. 2022-2023}

%\setlength\parindent{0pt}

\setcounter{tocdepth}{5}
\setcounter{secnumdepth}{5}

\lhead{\textbf{Document: }Report Finale}
\rhead{\textbf{Revision: }0.1}
\cfoot{\thepage /\pageref{LastPage}}

\hypersetup{
    colorlinks=true,
    linkcolor=blue,
    urlcolor=blue,
    pdftitle={Report Finale - Gruppo T56},
}

\pagestyle{fancy}
\newcommand{\nome}[0]{Life Planner\xspace}

\begin{document}

\begin{titlepage}
    \begin{figure}[!htb]
        \minipage{0.5\textwidth}
        \includegraphics[width=0.7\textwidth]{img/logo_unitn.png}
        \endminipage
        \hfill
        \minipage{0.5\textwidth}
        \begin{flushright}
            \Large
            Dipartimento d'Ingegneria e Scienze dell'informazione
        \end{flushright}
        \endminipage
        \hfill
    \end{figure}

    \vspace{6cm}

    \large
    \textbf{Progetto:}
    \begin{center}
        \Huge
        \color{blue}
        \textbf{Life planner}
    \end{center}

    \vspace{1cm}

    \textbf{Titolo del documento:}
    \begin{center}
        \huge
        \color{blue}
        \textbf{Documento di progetto}\\
        \textbf{(Obiettivi e Requisiti)}
    \end{center}


    
\end{titlepage}
\pagebreak

\tableofcontents
\pagebreak

\section{Organizzazione del lavoro}
\todo{completare}
Spiegare brevemente in questa sezione come è stato organizzato il lavoro. Ad esempio, indicare se c'è stata una suddivisione del lavoro per ruolo/competenze, che tipo d'interazioni tra i componenti del gruppo, quante volte ci si vedeva, quali strumenti sono stati utilizzati etc.

\section{Ruoli e attività}
\todo{completare}
\begin{center}
    \begin{tabular}{|c|p{0.2\linewidth}|p{0.3\linewidth}|}
        \hline
        Componente del team & Ruolo                                                  & Principali attività                                                                                                                                                                                       \\
        \hline
        Gabriele Lacchin    & Project leader / Analista / Progettista / Sviluppatore & Il ruolo principale è stato quello della gestione progetto, ….. Si è occupata inoltre della scrittura dei documenti e la loro sottomissione. Ha contributo a tutti i deliverable e in particolare D1 e D4 \\
        \hline
        Denis Lucietto      & Project leader / Analista / Progettista / Sviluppatore & Il ruolo principale è stato quello della gestione progetto, ….. Si è occupata inoltre della scrittura dei documenti e la loro sottomissione. Ha contributo a tutti i deliverable e in particolare D1 e D4 \\
        \hline
        Emanuele Zini       & Project leader / Analista / Progettista / Sviluppatore & Il ruolo principale è stato quello della gestione progetto, ….. Si è occupata inoltre della scrittura dei documenti e la loro sottomissione. Ha contributo a tutti i deliverable e in particolare D1 e D4 \\
        \hline
    \end{tabular}
\end{center}

\section{Carico e distribuzione del lavoro}
\todo{completare}
Di seguito il carico di lavoro espresso in ore/persona per ciascun membro del gruppo. Spiegare inoltre eventuali squilibri e/o commentare in base ai ruoli dei singoli componenti.

\begin{center}
    \begin{comment}
    \begin{tabular}{|c|c|c|c|c|c|c|}
        \hline
                         & D1   & D2   & D3   & D4 & D5 & TOT   \\
        \hline
        Gabriele Lacchin & 51.5 & 81   & 21   &    &    & 153.5 \\
        \hline
        Denis Lucietto   & 40   & 66.5 & 26.5 &    & 1  & 134   \\
        \hline
        Emanuele Zini    & 49.5 & 33.5 & 31   &    &    & 114   \\
        \hline
        Total            & 141  & 181  & 78.5 &    &    &       \\
        \hline
    \end{tabular}
    \end{comment}

    \begin{tblr}{colspec={lccccc|c},process=\funcSum}
                         & D1     & D2     & D3     & D4     & D5     & TOT    \\
        Gabriele Lacchin & 51     & 81     & 22     & 0      & 0      & rowsum \\
        Denis Lucietto   & 40     & 66     & 27     & 73     & 1      & rowsum \\
        Emanuele Zini    & 49     & 34     & 31     & 0      & 0      & rowsum \\
        \hline
        Total            & colsum & colsum & colsum & colsum & colsum          \\
    \end{tblr}
\end{center}



\section{Criticità}
\todo{completare}

Illustrare brevemente quali sono stati i problemi principali del progetto e come sono stati
risolti. Ad esempio “… inizialmente l'organizzazione del lavoro non era ben distribuito e tutti
facevano tutto non ottimizzando il tempo complessivo dedicato al D1. Questo è evidente anche
dalla tabella precedente. Successivamente nel D2 abbiamo assegnato dei ruoli precisi ad ogni
componente e ci siamo organizzati in maniera da lavorare indipendentemente incontrandoci
al termine di ciascuna lezione per fare il punto della situazione. Fabio Bianchi ha avuto dei
problemi personali nel periodo del D2 e non ha potuto lavorare al deliverable D2. C'è stato poi
nel D4 un parziale riequilibrio del tempo (come da tabella). Giovanna Verdi è stata quella che
ha lavorato di più al progetto sia in termini di tempo dedicato che come qualità del lavoro
svolto. E' stata un riferimento e una guida per tutti gli altri …… “

\section{Autovalutazione}
\todo{completare}
Nel complesso abbiamo lavorato tutti con impegno e costanza nell'arco di tutto il progetto. Ci
siamo resi conto che avremmo potuto far meglio nel D2. Giovanna Verdi è stata indubbiamente
la persona che si è dedicata di più al progetto mentre Fabio Bianchi spesso non è riuscito a
stare al passo con le tempistiche del progetto. Sulla base di queste considerazioni e della
qualità complessiva del lavoro svolto, la nostra autovalutazione è:

\begin{center}
    \begin{tabular}{|c|c|}
        \hline
                         & Voto \\
        \hline
        Gabriele Lacchin & 30   \\
        \hline
        Denis Lucietto   & 30   \\
        \hline
        Emanuele Zini    & 30   \\
        \hline
    \end{tabular}
\end{center}

\end{document}
